\documentclass[a4paper,12pt]{article}

%%% Работа с русским языком % для pdfLatex
\usepackage{cmap}					% поиск в~PDF
\usepackage{mathtext} 				% русские буквы в~фомулах
\usepackage[T2A]{fontenc}			% кодировка
\usepackage[utf8]{inputenc}			% кодировка исходного текста
\usepackage[english,russian]{babel}	% локализация и переносы
\usepackage{indentfirst} 			% отступ 1 абзаца

%%% Работа с русским языком % для XeLatex
%\usepackage[english,russian]{babel}   %% загружает пакет многоязыковой вёрстки
%\usepackage{fontspec}      %% подготавливает загрузку шрифтов Open Type, True Type и др.
%\defaultfontfeatures{Ligatures={TeX},Renderer=Basic}  %% свойства шрифтов по умолчанию
%\setmainfont[Ligatures={TeX,Historic}]{Times New Roman} %% задаёт основной шрифт документа
%\setsansfont{Comic Sans MS}                    %% задаёт шрифт без засечек
%\setmonofont{Courier New}
%\usepackage{indentfirst}
%\frenchspacing

%%% Дополнительная работа с математикой
\usepackage{amsfonts,amssymb,amsthm,mathtools}
\usepackage{amsmath}
\usepackage{icomma} % "Умная" запятая: $0,2$ --- число, $0, 2$ --- перечисление
\usepackage{upgreek}
%\usepackage{mathassents}

%% Номера формул
%\mathtoolsset{showonlyrefs=true} % Показывать номера только у тех формул, на которые есть \eqref{} в~тексте.

%%% Страница
\usepackage{extsizes} % Возможность сделать 14-й шрифт

%% Шрифты
\usepackage{euscript}	 % Шрифт Евклид
\usepackage{mathrsfs} % Красивый матшрифт

%% Свои команды
\DeclareMathOperator{\sgn}{\mathop{sgn}} % создание новой конанды \sgn (типо как \sin)
\usepackage{csquotes} % ещё одна штука для цитат
\newcommand{\pd}[2]{\ensuremath{\cfrac{\partial #1}{\partial #2}}} % частная производная
\newcommand{\abs}[1]{\ensuremath{\left|#1\right|}} % модуль
\renewcommand{\phi}{\ensuremath{\varphi}} % греческая фи
\newcommand{\pogk}[1]{\!\left(\cfrac{\sigma_{#1}}{#1}\right)^{\!\!\!2}\!} % для погрешностей

% Ссылки
\usepackage{color} % подключить пакет color
% выбрать цвета
\definecolor{BlueGreen}{RGB}{49,152,255}
\definecolor{Violet}{RGB}{120,80,120}
% назначить цвета при подключении hyperref
\usepackage[unicode, colorlinks, urlcolor=blue, linkcolor=blue, pagecolor=blue, citecolor=blue]{hyperref} %синие ссылки
%\usepackage[unicode, colorlinks, urlcolor=black, linkcolor=black, pagecolor=black, citecolor=black]{hyperref} % для печати (отключить верхний!)


%% Перенос знаков в~формулах (по Львовскому)
\newcommand*{\hm}[1]{#1\nobreak\discretionary{}
	{\hbox{$\mathsurround=0pt #1$}}{}}

%%% Работа с картинками
\usepackage{graphicx}  % Для вставки рисунков
\graphicspath{{images/}{images2/}}  % папки с картинками
\setlength\fboxsep{3pt} % Отступ рамки \fbox{} от рисунка
\setlength\fboxrule{1pt} % Толщина линий рамки \fbox{}
\usepackage{wrapfig} % Обтекание рисунков и таблиц текстом
\usepackage{multicol}

%%% Работа с таблицами
\usepackage{array,tabularx,tabulary,booktabs} % Дополнительная работа с таблицами
\usepackage{longtable}  % Длинные таблицы
\usepackage{multirow} % Слияние строк в~таблице
\usepackage{caption}
\captionsetup{labelsep=period, labelfont=bf}

%%% Оформление
\usepackage{indentfirst} % Красная строка
%\setlength{\parskip}{0.3cm} % отступы между абзацами
%%% Название разделов
\usepackage{titlesec}
\titlelabel{\thetitle.\quad}
\renewcommand{\figurename}{\textbf{Рис.}}		%Чтобы вместо figure под рисунками писал "рис"
\renewcommand{\tablename}{\textbf{Таблица}}		%Чтобы вместо table над таблицами писал Таблица

%%% Теоремы
\theoremstyle{plain} % Это стиль по умолчанию, его можно не переопределять.
\newtheorem{theorem}{Теорема}[section]
\newtheorem{proposition}[theorem]{Утверждение}

\theoremstyle{definition} % "Определение"
\newtheorem{definition}{Определение}[section]
\newtheorem{corollary}{Следствие}[theorem]
\newtheorem{problem}{Задача}[section]

\theoremstyle{remark} % "Примечание"
\newtheorem*{nonum}{Решение}
\newtheorem{zamech}{Замечание}[theorem]

%%% Правильные мат. символы для русского языка
\renewcommand{\epsilon}{\ensuremath{\varepsilon}}
\renewcommand{\phi}{\ensuremath{\varphi}}
\renewcommand{\kappa}{\ensuremath{\varkappa}}
\renewcommand{\le}{\ensuremath{\leqslant}}
\renewcommand{\leq}{\ensuremath{\leqslant}}
\renewcommand{\ge}{\ensuremath{\geqslant}}
\renewcommand{\geq}{\ensuremath{\geqslant}}
\renewcommand{\emptyset}{\varnothing}

\usepackage{bm} %жирный греческий шрифт
%\usepackage{ulem}

\graphicspath{{images}}


\usepackage{graphicx,xcolor,adjustbox,setspace, amsmath, hyperref}
\usepackage[normalem]{ulem} 

\newcommand{\resh}{\noindent\textit{Решение:}\\}

\newcounter{prim}
\newenvironment{prim}{%
	\addtocounter{prim}{1}
	\noindent{\\
		\textbf{\noindentПример \arabic{prim}\\}}%
}{\vspace{2mm}\\
\resh
}
\definecolor{orange}{rgb}{1, 0.7, 0.1}
%\usepackage{ulem}

\usepackage{bm} %жирный греческий шрифт

\newenvironment{psm}
{\left(\begin{smallmatrix*}[r]}
	{\end{smallmatrix*}\right)}

\newenvironment{pmatrixr}
{\begin{pmatrix*}[r]}
	{\end{pmatrix*}}

\newcounter{prob}
\newenvironment{prob}[1]{%
	\addtocounter{prob}{1}
	\noindent{\\
		\textbf{\noindent\arabic{prob}-{#1}}}%
}

\renewcommand{\figurename}{\textbf{Рис.}}		%Чтобы вместо figure под рисунками писал "рис"
\renewcommand{\tablename}{\textbf{Таблица}}		%Чтобы вместо table над таблицами писал Таблица


\title{statterm-3sem}
\author{Кожарин Алексей}
\date{November 2017}
\usepackage[left=1.27cm,right=1.27cm,top=1.27cm,bottom=2cm]{geometry}

\begin{document}
	
	\begin{center}
		\textbf{РЕШЕНИЯ ЗАДАЧ СЕМИНАРСКОЙ КОНТРОЛЬНОЙ ПО СТАТ. ТЕРМОДИНАМИКЕ}
	\end{center}
	%
	%	\hfill\begin{minipage}{0.4\textwidth}
	%		{\centering
	%			При поддержке:\\
	%			А. Кожарин, \href{https://vk.com/alekseik1}{VK}\\
	%			Г. Демьянов, \href{https://vk.com/id37346992}{VK}\\
	%			Г. Ильдар, \href{https://vk.com/gabdrakhmanovildar}{VK}\\
	%			В. Линева, \href{https://vk.com/vika_lineva}{VK}\\
	%		}
	%	\end{minipage}%
		
		\begin{center}
			{\textbf{Вариант 1}}
		\end{center}
		
	\begin{prob}{1}
		Получите выражение для частной производной $\left(\cfrac{\partial G}{\partial p}\right)_T$ как функции параметров состояния $p, V, T$ системы.
		\begin{nonum}
			$G=U + PV - TS; ~~ dG = -SdT + Vdp$ \\
			Отсюда, учитывая $dT = 0$, получим
			\begin{equation}
				\left(\cfrac{\partial G}{\partial p}\right)_T = V
			\end{equation}
		\end{nonum}
	\end{prob}
	
	\begin{prob}{1}
		Температурная зависимость теплоёмкости CH$_4$ задаётся формулой: $C_V = 9.1 + 6\cdot10^{-2}\cdot T - 10^{-6}\cdot T^2 - 7.2\cdot 10^{-9}\cdot T^3$. Определите колебательную составляющую теплоёмкости при температурах 298 К и 1200 К, а также предельную долю C$_\text{кол}$.
		\begin{nonum}
			$C_V = C_{V}^\text{\,вращ} + C_V^\text{\,пост} + C_V^\text{\,кол}$ \\
			Поскольку указанные температуры велики, вращательная теплоемкость постоянна и равна $C_V^\text{\,вращ} = \frac{3}{2}R = 12.465\, \frac{\text{Дж}}{\text{моль}\cdot\text{К}}$\,. Аналогично, для $C_V^\text{\,пост}= \frac{3}{2}R = 12.465\, \frac{\text{Дж}}{\text{моль}\cdot\text{К}}$\,. Тогда зависимость колебательной составляющей будет иметь вид: $C_V^\text{\,кол} = C_V - \frac{3}{2}R - \frac{3}{2}R$\,.
			\begin{equation}
				C_V^\text{\,кол} = -15.83 + 6\cdot10^{-2}\cdot T - 10^{-6}\cdot T^2 - 7.2\cdot 10^{-9}\cdot T^3
			\end{equation}\label{eq:2}
			Подставляя $T=298~K$ и $T = 1200~K$ в (2), получим:
			$C_V^\text{\,колеб} (298~K) = \boxed{1.77 \frac{\text{Дж}}{\text{моль}\cdot\text{К}}}$, $C_V^\text{\,колеб} (1000~K) = \boxed{35.96 \frac{\text{Дж}}{\text{моль}\cdot\text{К}}}$\,.
			
			Доля колебательной теплоемкости:
			$$
			\kappa_{298} = \cfrac{C_V^\text{кол}}{C_V} = 0.066 = \boxed{6.6\%}
			$$
			$$
			\kappa_{1000} = 0.59 = \boxed{59\%}
			$$
			
			В пределе:
			$$
			C_V = 3/2 R + 3/2 R + (3*N-6)R = 3R + 9R = 12R
			$$
			$$
			C_V^\text{кол} = (3N-6)R = 9R
			$$
			$$
			\boxed{\kappa_\text{кол}^\text{пред} = \cfrac{9R}{12R} = \cfrac{3}{4}}
			$$
			Видно, что $\kappa_{1000} > \kappa_{298}$, так как $\theta_\text{колеб} \sim 10^3~\text{K}$ и при этой температуре возбуждаются колебательные уровни энергии.
		\end{nonum}
	\end{prob}
	
	\begin{prob}{1}
		Какой из вращательных уровней молекулы О$_2$ наиболее заселён при температуре 1000 К? Момент инерции молекулы О$_2$ принять равным $19\cdot 10^{-47} \text{кг}\cdot \text{м}^2$.
		\begin{nonum}
			Характеристическая вращательная температура:
			$$
			\theta_\text{вр} = \cfrac{\hbar}{2J k} = \cfrac{1.054^2\cdot 10^{-68}}{2\cdot 19\cdot 10^{-47}\cdot 1.38\cdot 10^{-23}} = 2.1~\text{K}
			$$
			Запишем выражение для статсуммы:
			$$
			Z_\text{вращ} = \cfrac{1}{6}\, 	\sum_{j=0}^{\infty} (2j+1) \exp\left[-\frac{\theta_\text{вр}}{T} j(j+1)\right]
			$$
			Вероятность встретить частицу на уровне с энергией $\epsilon$:
			$$
			P(\epsilon_\text{вращ}) = \cfrac{\exp \left[-\frac{\theta_\text{вр}}{T} j(j+1)\right] (2j+1)}{Z_\text{вращ}}
			$$
			Найдем ее максимум:
			$$
			\cfrac{\partial}{\partial j}\, \left[(2j+1)\exp\left(-\frac{\theta_\text{вр}}{T} j (j+1)\right)\right] = 2e^x - (2j+1)^2 e^x \left(\cfrac{\theta_\text{вр}}{T}\right) = 0
			$$
			$$
			2 - \cfrac{\theta_\text{вр}}{T} (2j+1)^2 = 0
			$$
			$$
			2j+1 = \sqrt{\cfrac{2T}{\theta_\text{вр}}} = \sqrt{\cfrac{2\cdot 10^3}{2.1}}
			$$
			$$
			j = 14.8
			$$
			
			Среди кандидатов теперь $j=14$ и $j=15$. Подставляя каждый из них в уравнение вероятности, находим, что при $\boxed{j=15}$ вероятность будет максимальна.
		\end{nonum}
	\end{prob}
	
	\begin{prob}{1}
		В одной половине изолированного и разделённого перегородкой сосуда находится 2 моля идеального газа. Другая половина сосуда пустая. Чему равно изменение энтропии системы и тепловой эффект расширения газа после удаления перегородки?
		\begin{nonum}
			Будем считать, что наш газ идеален, а оболочка адиабатическая. При этом, в отсутствии теплообмена $dQ = 0$, а идеальный газ при расширении в вакуум \underline{не} совершает работы, следовательно, изменение его внутренней энергии равно нулю. Тепловой эффект, равный изменению энтальпии, тоже равен нулю.
		
			Разберемся с энтропией. По формуле (см. Кириченко "Термодинамика"):
			$$\Delta S = \nu R \ln\left(\cfrac{V_2}{V_1}\right) + \nu\, C_V \ln\left(\cfrac{T_2}{T_1}\right) $$
			
			Имеем: $\frac{V_2}{V_1} = 2$, то есть $\Delta S = \nu R \ln 2 = \boxed{11.52\, \frac{\text{Дж}}{\text{К}}}$
		\end{nonum}
	\end{prob}
	\setcounter{prob}{0}
	\vspace{0.5cm}
	\begin{center}
		\textbf{Вариант 2}
	\end{center}
	
	\begin{prob}{2}
		Получите выражение для частной производной $\left(\cfrac{\partial F}{\partial V}\right)_T$ как функции параметров состояния $p, V, T$ системы.
		\begin{nonum}
			$F = U - TS;~~ dF = -pdV - SdT \Rightarrow \boxed{\left(\cfrac{\partial F}{\partial V}\right)_T = -p}$
		\end{nonum}
	\end{prob}
	
	\begin{prob}{2}
		Температурная зависимость теплоёмкости CCl$_4$ задаётся формулой: $C_V = 89.3 + 9.6 \cdot 10^{-3}\,T -1.5\cdot 10^{-4}/T^2$. Определите колебательную составляющую теплоёмкости при температурах 298 К и 1200 К, а также предельную долю $C_\text{кол}$.
		\begin{nonum}
			$C_V = C_{V}^\text{\,вращ} + C_V^\text{\,пост} + C_V^\text{\,кол}$ \\
			Поскольку указанные температуры велики, вращательная теплоемкость постоянна и равна $C_V^\text{\,вращ} = \frac{3}{2}R = 12.465\, \frac{\text{Дж}}{\text{моль}\cdot\text{К}}$\,. Аналогично, для $C_V^\text{\,пост}= \frac{3}{2}R = 12.465\, \frac{\text{Дж}}{\text{моль}\cdot\text{К}}$\,. Тогда зависимость колебательной составляющей будет иметь вид: $C_V^\text{\,кол} = C_V - \frac{3}{2}R - \frac{3}{2}R$\,.
			$$
				C_V^{298} = 89.3 + 9.6\cdot 10^{-3}\cdot 298 - 1.5\cdot 10^{-4}/298^2 = 89.3 + 2.8608 - 1.7\cdot 10^{-9} = 92.1608 \frac{\text{Дж}}{\text{моль}\cdot\text{К}}
			$$
			$$
			C_\text{кол}^{298} = C_V^{298} - C_\text{пост} - C_\text{вр} = 92.1608 - 24.93 = 67.2301 \frac{\text{Дж}}{\text{моль}\cdot\text{К}}
			$$
			$$
			\boxed{\kappa_{298} = \cfrac{67.2308}{92.1608} = 0.7295}
			$$
			
			$$
			C_V^{1200} = 89.3 + 9.6\cdot 10^{-3}\cdot 1200 - 1.5\cdot 10^{-4}/1200^2 = 89.3 + 11.52 - 0.01\cdot 10^{-8} = 100.82 \frac{\text{Дж}}{\text{моль}\cdot\text{К}}
			$$
			$$
			C_\text{кол}^{1200} = C_V^{1200} - C_\text{пост} - C_\text{вр} = 100.82 - 24.93 = 75.89 \frac{\text{Дж}}{\text{моль}\cdot\text{К}}
			$$
			$$
			\boxed{\kappa_{1200} = \cfrac{75.89}{100.82} = 0.7527}
			$$
			
			В пределе:
			$$
			C_V = 3/2 R + 3/2 R + (3*N-6)R = 3R + 9R = 12R
			$$
			$$
			C_V^\text{кол} = (3N-6)R = 9R
			$$
			$$
			\boxed{\kappa_\text{кол}^\text{пред} = \cfrac{9R}{12R} = \cfrac{3}{4}}
			$$
		\end{nonum}
	\end{prob}
	
	\begin{prob}{2}
		Какой из вращательных уровней молекулы CО наиболее заселён при температуре 2000 К? Момент инерции молекулы СО принять равным $14.5\cdot10^{-47} \text{кг} \cdot \text{м}^2$.
		\begin{nonum}
			Характеристическая вращательная температура:
			$$
			\theta_\text{вр} = \cfrac{\hbar^2}{2J k} = \cfrac{1.054^2\cdot 10^{-68}}{2\cdot 14.5\cdot 10^{-47}\cdot 1.38\cdot 10^{-23}} = 2.78~\text{K}
			$$
			Запишем выражение для статсуммы:
			$$
			Z_\text{вращ} = \cfrac{1}{6}\, 	\sum_{j=0}^{\infty} (2j+1) \exp\left[-\frac{\theta_\text{вр}}{T} j(j+1)\right]
			$$
			Вероятность встретить частицу на уровне с энергией $\epsilon$:
			$$
			P(\epsilon_\text{вращ}) = \cfrac{\exp \left[-\frac{\theta_\text{вр}}{T} j(j+1)\right] (2j+1)}{Z_\text{вращ}}
			$$
			Найдем ее максимум:
			$$
			\cfrac{\partial}{\partial j}\, \left[(2j+1)\exp\left(-\frac{\theta_\text{вр}}{T} j (j+1)\right)\right] = 2e^x - (2j+1)^2 e^x \left(\cfrac{\theta_\text{вр}}{T}\right) = 0
			$$
			$$
			2 - \cfrac{\theta_\text{вр}}{T} (2j+1)^2 = 0
			$$
			$$
			2j+1 = \sqrt{\cfrac{2T}{\theta_\text{вр}}} = \sqrt{\cfrac{2\cdot 10^3}{2.1}}
			$$
			$$
			j = 18.5
			$$
			
			Среди кандидатов теперь $j=18$ и $j=19$. Подставляя каждый из них в уравнение вероятности, находим, что при $\boxed{j=18}$ вероятность будет максимальна.
		\end{nonum}
	\end{prob}
	
	\begin{prob}{2}
		Произведения главных моментов инерции для молекул бромбензола и пара-дихлорбензола практически одинаковы. Какова разность мольных вращательных энтропий этих газов?
		\begin{nonum}
			По формуле для энтропии:
			$$
			S_\text{вр} = \cfrac{1}{2} R \ln I_1 I_2 I_3 + \cfrac{3}{2} R \ln T - R \ln \sigma + 1320.8 ~~\left[\cfrac{\text{Дж}}{\text{моль}\cdot\text{К}}\right]
			$$
			
			При прочих равных параметрах разность создаст только число симметрии, которое для бромбензола равно $\sigma_1 = 2$, а для п-дихлорбезнола $\sigma = 4$. Тогда $\Delta S_\text{вр} = -R \ln \sigma_2 + R\ln \sigma_1 = R\ln \cfrac{4}{2}$
			$$
			\boxed{\Delta S_\text{вр}  = 5.76 \cfrac{\text{Дж}}{\text{моль}\cdot\text{К}}}
			$$
		\end{nonum}
	\end{prob}
	
\end{document}
