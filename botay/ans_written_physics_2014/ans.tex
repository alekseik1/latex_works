\documentclass[a4paper,12pt]{article}

%%% Работа с русским языком % для pdfLatex
\usepackage{cmap}					% поиск в~PDF
\usepackage{mathtext} 				% русские буквы в~фомулах
\usepackage[T2A]{fontenc}			% кодировка
\usepackage[utf8]{inputenc}			% кодировка исходного текста
\usepackage[english,russian]{babel}	% локализация и переносы
\usepackage{indentfirst} 			% отступ 1 абзаца

%%% Работа с русским языком % для XeLatex
%\usepackage[english,russian]{babel}   %% загружает пакет многоязыковой вёрстки
%\usepackage{fontspec}      %% подготавливает загрузку шрифтов Open Type, True Type и др.
%\defaultfontfeatures{Ligatures={TeX},Renderer=Basic}  %% свойства шрифтов по умолчанию
%\setmainfont[Ligatures={TeX,Historic}]{Times New Roman} %% задаёт основной шрифт документа
%\setsansfont{Comic Sans MS}                    %% задаёт шрифт без засечек
%\setmonofont{Courier New}
%\usepackage{indentfirst}
%\frenchspacing

%%% Дополнительная работа с математикой
\usepackage{amsfonts,amssymb,amsthm,mathtools}
\usepackage{amsmath}
\usepackage{icomma} % "Умная" запятая: $0,2$ --- число, $0, 2$ --- перечисление
\usepackage{upgreek}
%\usepackage{mathassents}

%% Номера формул
%\mathtoolsset{showonlyrefs=true} % Показывать номера только у тех формул, на которые есть \eqref{} в~тексте.

%%% Страница
\usepackage{extsizes} % Возможность сделать 14-й шрифт

%% Шрифты
\usepackage{euscript}	 % Шрифт Евклид
\usepackage{mathrsfs} % Красивый матшрифт

%% Свои команды
\DeclareMathOperator{\sgn}{\mathop{sgn}} % создание новой конанды \sgn (типо как \sin)
\usepackage{csquotes} % ещё одна штука для цитат
\newcommand{\pd}[2]{\ensuremath{\cfrac{\partial #1}{\partial #2}}} % частная производная
\newcommand{\abs}[1]{\ensuremath{\left|#1\right|}} % модуль
\renewcommand{\phi}{\ensuremath{\varphi}} % греческая фи
\newcommand{\pogk}[1]{\!\left(\cfrac{\sigma_{#1}}{#1}\right)^{\!\!\!2}\!} % для погрешностей

% Ссылки
\usepackage{color} % подключить пакет color
% выбрать цвета
\definecolor{BlueGreen}{RGB}{49,152,255}
\definecolor{Violet}{RGB}{120,80,120}
% назначить цвета при подключении hyperref
\usepackage[unicode, colorlinks, urlcolor=blue, linkcolor=blue, pagecolor=blue, citecolor=blue]{hyperref} %синие ссылки
%\usepackage[unicode, colorlinks, urlcolor=black, linkcolor=black, pagecolor=black, citecolor=black]{hyperref} % для печати (отключить верхний!)


%% Перенос знаков в~формулах (по Львовскому)
\newcommand*{\hm}[1]{#1\nobreak\discretionary{}
	{\hbox{$\mathsurround=0pt #1$}}{}}

%%% Работа с картинками
\usepackage{graphicx}  % Для вставки рисунков
\graphicspath{{images/}{images2/}}  % папки с картинками
\setlength\fboxsep{3pt} % Отступ рамки \fbox{} от рисунка
\setlength\fboxrule{1pt} % Толщина линий рамки \fbox{}
\usepackage{wrapfig} % Обтекание рисунков и таблиц текстом
\usepackage{multicol}

%%% Работа с таблицами
\usepackage{array,tabularx,tabulary,booktabs} % Дополнительная работа с таблицами
\usepackage{longtable}  % Длинные таблицы
\usepackage{multirow} % Слияние строк в~таблице
\usepackage{caption}
\captionsetup{labelsep=period, labelfont=bf}

%%% Оформление
\usepackage{indentfirst} % Красная строка
%\setlength{\parskip}{0.3cm} % отступы между абзацами
%%% Название разделов
\usepackage{titlesec}
\titlelabel{\thetitle.\quad}
\renewcommand{\figurename}{\textbf{Рис.}}		%Чтобы вместо figure под рисунками писал "рис"
\renewcommand{\tablename}{\textbf{Таблица}}		%Чтобы вместо table над таблицами писал Таблица

%%% Теоремы
\theoremstyle{plain} % Это стиль по умолчанию, его можно не переопределять.
\newtheorem{theorem}{Теорема}[section]
\newtheorem{proposition}[theorem]{Утверждение}

\theoremstyle{definition} % "Определение"
\newtheorem{definition}{Определение}[section]
\newtheorem{corollary}{Следствие}[theorem]
\newtheorem{problem}{Задача}[section]

\theoremstyle{remark} % "Примечание"
\newtheorem*{nonum}{Решение}
\newtheorem{zamech}{Замечание}[theorem]

%%% Правильные мат. символы для русского языка
\renewcommand{\epsilon}{\ensuremath{\varepsilon}}
\renewcommand{\phi}{\ensuremath{\varphi}}
\renewcommand{\kappa}{\ensuremath{\varkappa}}
\renewcommand{\le}{\ensuremath{\leqslant}}
\renewcommand{\leq}{\ensuremath{\leqslant}}
\renewcommand{\ge}{\ensuremath{\geqslant}}
\renewcommand{\geq}{\ensuremath{\geqslant}}
\renewcommand{\emptyset}{\varnothing}

\usepackage{bm} %жирный греческий шрифт
%\usepackage{ulem}

\graphicspath{{images}}


\usepackage{graphicx,xcolor,adjustbox,setspace, amsmath}

\newcommand{\resh}{\noindent\textit{Решение:}\\}

\newcounter{prim}
\newenvironment{prim}{%
	\addtocounter{prim}{1}
	\noindent{\\
		\textbf{\noindentПример \arabic{prim}\\}}%
}{\vspace{2mm}\\
\resh
}
\definecolor{orange}{rgb}{1, 0.7, 0.1}
%\usepackage{ulem}

\usepackage{bm} %жирный греческий шрифт

\newenvironment{psm}
{\left(\begin{smallmatrix*}[r]}
	{\end{smallmatrix*}\right)}

\newenvironment{pmatrixr}
{\begin{pmatrix*}[r]}
	{\end{pmatrix*}}

\newcounter{prob}
\newenvironment{prob}{%
	\addtocounter{prob}{1}
	\noindent{\\
		\textbf{\noindent\arabic{prob}A}}%
}

\renewcommand{\figurename}{\textbf{Рис.}}		%Чтобы вместо figure под рисунками писал "рис"
\renewcommand{\tablename}{\textbf{Таблица}}		%Чтобы вместо table над таблицами писал Таблица


\title{ans2014}
\author{Кожарин Алексей}
\date{June 2017}
\usepackage[left=1.27cm,right=1.27cm,top=1.27cm,bottom=2cm]{geometry}

\begin{document}
	
	\begin{center}
		\textbf{РЕШЕНИЯ КОНТРОЛЬНОЙ 2 июня 2014 г}
	\end{center}
	
	\begin{prob}
		(Булыгин В.С.) $C(P) = \nu (C_P - RT/(P\frac{dT}{dP}))$ (см.
		\href{https://mipt.ru/education/chair/physics/S_II/method/C_PVT.pdf}
		{Методические материалы 2 сем.}). Так как $T(P) = T_1 +
		\cfrac{T_2-T_1}{P_2-P_1}(P-P_1)$ (лин. зависимость), то с учётом
		соотношения Майера и при $C_V = \cfrac{3}{2} R$: \\
		$C(P) = \nu\left(C_V - \cfrac{R}{P}\ \cfrac{T_1 P_2 - P_1 T_2}
		{T_2 - T_1} \right) = \nu\,C_V \left(1 - \cfrac{2}{3}\ \cfrac
		{T_1 P_2 - P_1 T_2}{T_2 - T_1}\ \cfrac{1}{P}\right)$,
		откуда $\partial(P) \equiv \cfrac{C(P)}{\nu C_V} = 1 - \cfrac{4}{3}\,
		\cfrac{1}{P} $. Из $\partial(P_2) = \cfrac{1}{3}, \partial(P_1) =
		- \cfrac{1}{3}$ следует ответ: \fbox{$\cfrac{C(P_2)}{C(P_1)} = 
			\cfrac{\partial(P_2)}{\partial(P_1)} = -1$} 
	\end{prob}
	
	\begin{prob}
		(Попов П.В.) Распределение плотности во вращающемся цилиндре 
		$\rho(r) = \rho_0\exp(\alpha r^2/a^2)$, где $\alpha = 
		\mu \omega^2 a^2/(2RT_0) \approx 10^{-2}$. Энергия, запасенная
		во вращении (в расчёте на 1 моль):
		$$
		E = \cfrac{\int\limits_{0}^{a} \cfrac{1}{2}\, \omega^2 r^2 
			\cdot \rho(r) \cdot 2 \pi r dr}{\int\limits_{0}^{a}
			\cfrac{\rho(r)}{\mu} \cdot 2\pi r dr} = RT_0\,\cfrac
		{1-(1+\alpha)e^{-\alpha}}{1-e^{-\alpha}} \approx \cfrac{1}{2}\,
		\alpha RT_0
		$$
		(то же самое получим, сразу пользуясь приближением $\rho(r)
		\approx \rho_0$)~: $$E\approx \cfrac{\int\limits_{0}^{a} \cfrac{1}{2}\,
			\mu \omega^2 r^2 \cdot \rho_0 \cdot 2\pi r dr}{\int\limits_{0}^
			{a} \rho_0 \cdot 2\pi r dr} = \cfrac{1}{2}\,\alpha RT_0$$
		Температура вырастет на $\Delta T = \cfrac{E}{C_V} = \cfrac{1}{2}\,
		\alpha \cfrac{RT_0}{C_V}\,=\cfrac{1}{5}\,\alpha T_0= \cfrac{\mu 
			\omega^2 a^2}{10R}\, \approx 270 \cdot \cfrac{10^{-2}}{5}
		\approx$ \fbox{$0,54K$}
	\end{prob}
	
	\begin{prob}
		(Прут Э.В.) Характеристические температуры $\theta_1 = \cfrac
		{h\nu_1}{k} \approx 480 K;~\theta_2 \approx 4320K; \theta_3 
		\approx 4800 K$. Колебательная теплоёмкость (см. задание 8.52
		из сборника)
		$$ C_{V}^{\text{~колеб}} = R \cfrac{\left(\frac{\theta}{T}
			\right)^2 \exp\left(\frac{\theta}{T}\right)}{\left(\exp\left(
			\frac{\theta}{T}\right)-1\right)^2} \underset{T < \theta}
		{\approx} R \left(\cfrac{\theta}{T}\right)^2 \exp\left(-\cfrac
		{\theta}{T}\right)
		$$
		\begin{tabular}{ll}
			При $T=T_1$: & При $T=T_2$: \\
			~~$T_1 < \theta_1$ \hspace*{0.2cm} $\Rightarrow C_{V}^{\,
				\text{колеб}}(\theta_1) \approx (4.8)^2 e^{-4.8}R \approx
			0.2R$\hspace{1cm} & ~~$T_2 > \theta_1$ \hspace*{0.2cm} $\Rightarrow C_{V}^{\,
				\text{колеб}}(\theta_1) \approx R$  \\
			~~$T_1 \ll \theta_2$ \hspace*{0.2cm} $\Rightarrow C_{V}^{\,
				\text{колеб}}(\theta_2) \approx 0$\hspace{1cm} & 
			~~$T_2 < \theta_2$ \hspace*{0.2cm} $\Rightarrow C_{V}^{\,
				\text{колеб}}(\theta_2) \approx (4.3)^2 e^{-4.3}R
			\approx 0.25R$  \\
			~~$T_1 \ll \theta_3$ \hspace*{0.2cm} $\Rightarrow C_{V}^{\,
				\text{колеб}}(\theta_3) \approx 0$\hspace{1cm} & 
			~~$T_2 < \theta_3$ \hspace*{0.2cm} $\Rightarrow C_{V}^{\,
				\text{колеб}}(\theta_3) \approx (4.8)^2 e^{-4.8}R
			\approx 0.2R$  \\
		\end{tabular}
		Итого: $(C_{V})_1 = 3R + 0.2R = 3.2R,~ (C_P)_1 = R + (C_V)_1
		 = 4.2R, \gamma_1 \approx 1.31, (C_V)_2 = 3R + R + 0.25R + 0.2R = 4.45R, (C_P)_2 = 5.45R,
		 \gamma_2 \approx 1.23$, \fbox{$\cfrac{\gamma_2}{\gamma_1} =
		 	\cfrac{1.23}{1.31} \approx 0.94$}
	\end{prob}
	
	\begin{prob}
		(Кириченко Н.А.) Смещение частицы между столкновениями с другими
		частицами $l \sim \frac{1}{n\sigma}, \\ \sigma \propto R^2$. При
		броуновском движении частица смещается на расстояние $l$ за время
		$t \sim l^2/D, \\D = kTB, B=\cfrac{1}{6\pi\eta R}$. Тогда $ t
		\sim \cfrac{1}{(n\sigma)^2}\, \cfrac{6\pi\eta R}{kT} \propto
		\cfrac{1}{n^2 R^3}$. Одна частица испытывает в единицу времени
		число столкновений $\nu \sim 1/t \propto n^2 R^3$, а $n$ частиц
		-- $v = \nu n \propto n^3 R^3$. С учётом сохранения массы:
		$n_0 R_0^3 = n_1 R_1^3$, получим \fbox{$\cfrac{v_1}{v_0} = 
			\cfrac{n_1^3 R_1^3}{n_0^3 R_0^3} = \cfrac{n_1^2}{n_0^2} = 4$}~.
	\end{prob}
	
	\begin{prob}
		(Коротков П.Ф.) Закон сохранения массы~ $\Pi_0 \rho_0 v_0 = \Pi
		\rho v$. Уравнение адиабаты $\cfrac{\rho}{\rho_0} = \left(\cfrac{T}
		{T_0}\right)^{\cfrac{1}{\gamma-1}}$. Преобразуем уравнение
		Бернулли $\frac{1}{2} \mu v^2 + C_P T = \const$ с учётом, что
		$v = M \sqrt{\frac{\gamma RT}{\mu}}; C_P = \frac{\gamma}{\gamma-1}
		R;$ \\ получим $T\left(1+\frac{\gamma - 1}{2} M^2 \right) = \const$.	% Тут у меня не получилось, нужно проверить потом
		Тогда
		$$
		\cfrac{\Pi}{\ \Pi_0} = \cfrac{\rho_0 v_0}{\rho v} = \cfrac{M_0}{M}
		\left(\cfrac{T_0}{T}\right)^\frac{\gamma+1}{2(\gamma-1)} = \cfrac{M_0}{M}
		\left(\cfrac{1+\frac{\gamma-1}{2}M^2}{1+\frac{\gamma-1}{2}M_0^2}\right)^
		\frac{\gamma+1}{2(\gamma-1)}
		$$ \\
		При $\gamma = 7/5, M = 2, M_0 = 1$ получим \fbox{$\cfrac{\Pi}{\ 
				\Pi_0} = \cfrac{27}{16} \approx 1.68$}~.
	\end{prob}

\end{document}
