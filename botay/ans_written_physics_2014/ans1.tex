\documentclass[a4paper,12pt]{article}

\input{preambula}


\usepackage{graphicx,xcolor,adjustbox,setspace, amsmath}

\newcommand{\resh}{\noindent\textit{Решение:}\\}

\newcounter{prim}
\newenvironment{prim}{%
	\addtocounter{prim}{1}
	\noindent{\\
		\textbf{\noindentПример \arabic{prim}\\}}%
}{\vspace{2mm}\\
\resh
}
\definecolor{orange}{rgb}{1, 0.7, 0.1}
%\usepackage{ulem}

\usepackage{bm} %жирный греческий шрифт

\newenvironment{psm}
{\left(\begin{smallmatrix*}[r]}
	{\end{smallmatrix*}\right)}

\newenvironment{pmatrixr}
{\begin{pmatrix*}[r]}
	{\end{pmatrix*}}

\newcounter{prob}
\newenvironment{prob}{%
	\addtocounter{prob}{1}
	\noindent{\\
		\textbf{\noindent\arabic{prob}Б}}%
}

\renewcommand{\figurename}{\textbf{Рис.}}		%Чтобы вместо figure под рисунками писал "рис"
\renewcommand{\tablename}{\textbf{Таблица}}		%Чтобы вместо table над таблицами писал Таблица


\title{ans2014}
\author{Кожарин Алексей}
\date{June 2017}
\usepackage[left=1.27cm,right=1.27cm,top=1.27cm,bottom=2cm]{geometry}

\begin{document}
	
	\begin{center}
		\textbf{РЕШЕНИЯ КОНТРОЛЬНОЙ 2 июня 2014 г}
	\end{center}
	
	\begin{prob}
		(Булыгин В.С.) $C(V) = \nu (C_V + RT/(V\frac{dT}{dV}))$ (см.
		\href{https://mipt.ru/education/chair/physics/S_II/method/C_PVT.pdf}
		{Методические материалы 2 сем.}). Так как $T(V) = T_1 +
		\cfrac{T_2-T_1}{V_2-V_1}(V-V_1)$ (лин. зависимость), то с учётом
		соотношения Майера и при $C_P = \cfrac{5}{2} R$: \\
		$C(V) = \nu\left(C_P - \cfrac{R}{V}\ \cfrac{T_1 V_2 - V_1 T_2}
		{T_2 - T_1} \right) = \nu\,C_P \left(1 - \cfrac{2}{5}\ \cfrac
		{T_1 V_2 - V_1 T_2}{T_2 - T_1}\ \cfrac{1}{V}\right)$,
		откуда $\partial(V) \equiv \cfrac{C(V)}{\nu C_P} = 1 - \cfrac{6}{5}\,
		\cfrac{1}{V} $. Из $\partial(V_2) = \cfrac{2}{5}, \partial(V_1) =
		- \cfrac{1}{5}$ следует ответ: \fbox{$\cfrac{C(V_2)}{C(V_1)} = 
			\cfrac{\partial(V_2)}{\partial(V_1)} = -2$} 
	\end{prob}
	
	\begin{prob}
		(Попов П.В.) Конечное распределение плотности в цилиндре 
		$\rho(x) = \rho_0\exp(\alpha x/l)$, где\\ $\alpha = 
		\mu al/(RT) \approx 10^{-2}$. В системе сосуда изменение
		внутренней энергии газа равно работе сил инерции: $\Delta E = 
		-M a(x_c - l/2)$, где $M$ -- масса газа, $x_c$ -- положение
		центра масс газа после установления равновесия:
		$$ x_c = \cfrac{\int\limits_{0}^{l} x\rho (x) dx}{\int\limits_{0}
			^{l}\rho (x) dx} = l\,\cfrac{1- (1+\alpha)e^{-\alpha}}{\alpha
			(1 - e^{-\alpha})} \approx \frac{1}{2}\,l - \frac{1}{12}\,\alpha
		 l
		$$ \\
		Исходная температура \underline{меньше} конечной на $\Delta T = 
		\cfrac{\Delta E}{\nu\,C_V} = \cfrac{\mu a}{\frac{5}{2}R}(l/2 
		- x_c) \approx \cfrac{1}{30} \alpha^2 T \approx$ \fbox{$10^{-3}K$}
	\end{prob}
	
	\begin{prob}
		(Прут Э.В.) Характеристические температуры $\theta_1 = \cfrac
		{h\nu_1}{k} \approx 192 K;~\theta_2 \approx 240K; \theta_3 
		\approx 3360 K$. Колебательная теплоёмкость (см. задание 8.52
		из сборника)
		$$ C_{V}^{\text{~колеб}} = R \cfrac{\left(\frac{\theta}{T}
			\right)^2 \exp\left(\frac{\theta}{T}\right)}{\left(\exp\left(
			\frac{\theta}{T}\right)-1\right)^2} \underset{T < \theta}
		{\approx} R \left(\cfrac{\theta}{T}\right)^2 \exp\left(-\cfrac
		{\theta}{T}\right)
		$$
		\begin{tabular}{ll}
			При $T=T_1$: & При $T=T_2$: \\
			~~$T_1 > \theta_1$ \hspace*{0.2cm} $\Rightarrow C_{V}^{\,
				\text{колеб}}(\theta_1) \approx R$\hspace{1cm} & ~~$T_2 
			\gg \theta_1$ \hspace*{0.2cm} $\Rightarrow C_{V}^{\,
				\text{колеб}}(\theta_1) \approx R$  \\
			~~$T_1 > \theta_2$ \hspace*{0.2cm} $\Rightarrow C_{V}^{\,
				\text{колеб}}(\theta_2) \approx R$\hspace{1cm} & 
			~~$T_2 \gg \theta_2$ \hspace*{0.2cm} $\Rightarrow C_{V}^{\,
				\text{колеб}}(\theta_2) \approx R$  \\
			~~$T_1 \ll \theta_3$ \hspace*{0.2cm} $\Rightarrow C_{V}^{\,
				\text{колеб}}(\theta_3) \approx 0$\hspace{1cm} & 
			~~$T_2 < \theta_3$ \hspace*{0.2cm} $\Rightarrow C_{V}^{\,
				\text{колеб}}(\theta_3) \approx \frac{(2.24)^2 e^{2.24}}
				{(e^{2.24}-1)^2}
			\approx 0.7R$
		\end{tabular}\\
		Итого: $(C_{V})_1 = 3R + 2R = 5R,~ (C_P)_1 = R + (C_V)_1
		 = 6R, \gamma_1 \approx 1.20, (C_V)_2 = 3R + R + R + 0.7R = 5.7R,
		  (C_P)_2 = 6.7R,
		 \gamma_2 \approx 1.175$, \fbox{$\cfrac{\gamma_2}{\gamma_1} =
		 	\cfrac{1.175}{1.20} \approx 0.98$}
	\end{prob}
	
	\begin{prob}
		(Кириченко Н.А.) Смещение частицы между столкновениями с другими
		частицами $l \sim \frac{1}{n\sigma}, \\ \sigma \propto R^2$. При
		броуновском движении частица смещается на расстояние $l$ за время
		$t \sim l^2/D, \\D = kTB, B=\cfrac{1}{6\pi\eta R}$. Тогда $ t
		\sim \cfrac{1}{(n\sigma)^2}\, \cfrac{6\pi\eta R}{kT} \propto
		\cfrac{1}{n^2 R^3}$. Одна частица испытывает в единицу времени
		число столкновений $\nu \sim 1/t \propto n^2 R^3$, а $n$ частиц
		-- $v = \nu n \propto n^3 R^3$. С учётом сохранения массы:
		$n_0 R_0^3 = n_1 R_1^3$, получим $\cfrac{v_1}{v_0} = 
			\cfrac{n_1^3 R_1^3}{n_0^3 R_0^3} = \cfrac{n_1^2}{n_0^2}
			\Rightarrow$ \fbox{$n_1 = n_0 \sqrt{\cfrac{\nu_1}{\nu_0}} =
				\cfrac{n_0}{2}$}.
	\end{prob}
	
	\begin{prob}
		(Коротков П.Ф.) Закон сохранения массы~ $\Pi_0 \rho_0 v_0 = \Pi
		\rho v$. Уравнение адиабаты $\cfrac{\rho}{\rho_0} = \left(\cfrac{T}
		{T_0}\right)^{\cfrac{1}{\gamma-1}}$. Преобразуем уравнение
		Бернулли $\frac{1}{2} \mu v^2 + C_P T = \const$ с учётом, что
		$v = M \sqrt{\frac{\gamma RT}{\mu}}; C_P = \frac{\gamma}{\gamma-1}
		R;$ \\ получим $T\left(1+\frac{\gamma - 1}{2} M^2 \right) = \const$.	% Тут у меня не получилось, нужно проверить потом
		Тогда
		$$
		\cfrac{\Pi}{\ \Pi_0} = \cfrac{\rho_0 v_0}{\rho v} = \cfrac{M_0}{M}
		\left(\cfrac{T_0}{T}\right)^\frac{\gamma+1}{2(\gamma-1)} = \cfrac{M_0}{M}
		\left(\cfrac{1+\frac{\gamma-1}{2}M^2}{1+\frac{\gamma-1}{2}M_0^2}\right)^
		\frac{\gamma+1}{2(\gamma-1)}
		$$ \\
		При $\gamma = 5/3, M = 1/2, M_0 = 1$ получим \fbox{$\cfrac{\Pi}{\ 
				\Pi_0} = \cfrac{64}{49} \approx 1.31$}~.
	\end{prob}

\end{document}
