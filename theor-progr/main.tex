\documentclass[a4paper,12pt]{article}
\include{preambula}

\renewcommand{\baselinestretch}{1.3}

\title{programm}
\date{\today}
\usepackage[left=1.27cm,right=1.27cm,top=1.27cm,bottom=2cm]{geometry}

\begin{document}
	\begin{center}
		\large\textbf{Программа экзамена по дисциплине "Теоретическая механика"}
	\end{center}
	\begin{enumerate}
	\item Понятие положения равновесия. Критерий положения равновесия стационарной системы в терминах обобщенных сил. Случай потенциальных сил. Принцип  виртуальных перемещений.
	
	\item Теорема  Лагранжа-Дирихле  об  устойчивости положения равновесия консервативной  системы. Теоремы Ляпунова и Четаева о неустойчивости положения равновесия консервативной системы (без доказательства).
	
	\item Влияние дополнительных гироскопических и диссипативных сил на устойчивость консервативной системы. Обобщение теоремы Лагранжа-Дирихле при действии диссипативных сил с полной диссипацией.
	
	\item Теорема Ляпунова об устойчивости и неустойчивости по линейному приближению. 
	
	\item Необходимые условия устойчивости полинома. Критерий  Рауса-Гурвица (без доказательства).
	
	\item Теоремы второго метода Ляпунова об устойчивости, неустойчивости и асимптотической устойчивости.
	
	\item Теорема Четаева о неустойчивости.
	
	\item Теорема Барбашина-Красовского.
	
	\item Понятия о бифуркациях, дивергенции и флаттере. 
	
	\item Малые колебания консервативной системы в окрестности устойчивого положения  равновесия. Использование симметрии системы для нахождения мод колебаний.
	
	\item Свойства амплитудных векторов в задаче о малых колебаниях консервативной системы, главные (нормальные) координаты.
	
	\item Действие внешней силы на линейную систему, движущуюся вблизи положения равновесия. Частотные характеристики.
	
	\item Канонические  уравнения  Гамильтона.  Физический смысл функции Гамильтона в случае консервативной системы.
	
	\item Первые интегралы уравнений движения. Критерий первого интеграла, скобки Пуассона и их свойства.
	
	
	\item Понижение порядка гамильтоновой системы при наличии первых интегралов. Уравнения Уиттекера.
	
	\item Циклические координаты. Обобщённо консервативные системы. Теорема Якоби-Пуассона.
	
	\item Действие по Гамильтону. Вариация  действия по Гамильтону в задаче с подвижными концами.
	
	
	\item Принцип Гамильтона.
	
	\item Ковариантность уравнений Лагранжа при замене координат и времени.
	
	\item Теорема Э. Нетер.
	
	\item Интегральные  инварианты Пуанкаре-Картана и Пуанкаре.
	
	\item Обратные  теоремы  теории  интегральных  инвариантов.
	
	\item Теорема Ли-Хуачжуна о множестве интегральных инвариантах первого порядка.
	
	\item Теорема Лиувилля: инвариантность фазового объема в системах с нулевой дивергенцией. Сохранение фазового объема в гамильтоновых системах.
	
	\item Канонические преобразования. Критерий каноничности преобразований.
	
	\item Свободные канонические преобразования.
	
	\item Преобразования функции Гамильтона при канонических преобразованиях.
	
	\item Уравнение Гамильтона-Якоби. Полный интеграл уравнения Гамильтона-Якоби. Разделение переменных.
\end{enumerate}
\end{document}