\documentclass[a4paper,12pt]{article}
%%% Работа с русским языком % для pdfLatex
\usepackage{cmap}					% поиск в~PDF
\usepackage{mathtext} 				% русские буквы в~фомулах
\usepackage[T2A]{fontenc}			% кодировка
\usepackage[utf8]{inputenc}			% кодировка исходного текста
\usepackage[english,russian]{babel}	% локализация и переносы
\usepackage{indentfirst} 			% отступ 1 абзаца

%%% Работа с русским языком % для XeLatex
%\usepackage[english,russian]{babel}   %% загружает пакет многоязыковой вёрстки
%\usepackage{fontspec}      %% подготавливает загрузку шрифтов Open Type, True Type и др.
%\defaultfontfeatures{Ligatures={TeX},Renderer=Basic}  %% свойства шрифтов по умолчанию
%\setmainfont[Ligatures={TeX,Historic}]{Times New Roman} %% задаёт основной шрифт документа
%\setsansfont{Comic Sans MS}                    %% задаёт шрифт без засечек
%\setmonofont{Courier New}
%\usepackage{indentfirst}
%\frenchspacing

%%% Дополнительная работа с математикой
\usepackage{amsfonts,amssymb,amsthm,mathtools}
\usepackage{amsmath}
\usepackage{icomma} % "Умная" запятая: $0,2$ --- число, $0, 2$ --- перечисление
\usepackage{upgreek}
%\usepackage{mathassents}

%% Номера формул
%\mathtoolsset{showonlyrefs=true} % Показывать номера только у тех формул, на которые есть \eqref{} в~тексте.

%%% Страница
\usepackage{extsizes} % Возможность сделать 14-й шрифт

%% Шрифты
\usepackage{euscript}	 % Шрифт Евклид
\usepackage{mathrsfs} % Красивый матшрифт

%% Свои команды
\DeclareMathOperator{\sgn}{\mathop{sgn}} % создание новой конанды \sgn (типо как \sin)
\usepackage{csquotes} % ещё одна штука для цитат
\newcommand{\pd}[2]{\ensuremath{\cfrac{\partial #1}{\partial #2}}} % частная производная
\newcommand{\abs}[1]{\ensuremath{\left|#1\right|}} % модуль
\renewcommand{\phi}{\ensuremath{\varphi}} % греческая фи
\newcommand{\pogk}[1]{\!\left(\cfrac{\sigma_{#1}}{#1}\right)^{\!\!\!2}\!} % для погрешностей

% Ссылки
\usepackage{color} % подключить пакет color
% выбрать цвета
\definecolor{BlueGreen}{RGB}{49,152,255}
\definecolor{Violet}{RGB}{120,80,120}
% назначить цвета при подключении hyperref
\usepackage[unicode, colorlinks, urlcolor=blue, linkcolor=blue, pagecolor=blue, citecolor=blue]{hyperref} %синие ссылки
%\usepackage[unicode, colorlinks, urlcolor=black, linkcolor=black, pagecolor=black, citecolor=black]{hyperref} % для печати (отключить верхний!)


%% Перенос знаков в~формулах (по Львовскому)
\newcommand*{\hm}[1]{#1\nobreak\discretionary{}
	{\hbox{$\mathsurround=0pt #1$}}{}}

%%% Работа с картинками
\usepackage{graphicx}  % Для вставки рисунков
\graphicspath{{images/}{images2/}}  % папки с картинками
\setlength\fboxsep{3pt} % Отступ рамки \fbox{} от рисунка
\setlength\fboxrule{1pt} % Толщина линий рамки \fbox{}
\usepackage{wrapfig} % Обтекание рисунков и таблиц текстом
\usepackage{multicol}

%%% Работа с таблицами
\usepackage{array,tabularx,tabulary,booktabs} % Дополнительная работа с таблицами
\usepackage{longtable}  % Длинные таблицы
\usepackage{multirow} % Слияние строк в~таблице
\usepackage{caption}
\captionsetup{labelsep=period, labelfont=bf}

%%% Оформление
\usepackage{indentfirst} % Красная строка
%\setlength{\parskip}{0.3cm} % отступы между абзацами
%%% Название разделов
\usepackage{titlesec}
\titlelabel{\thetitle.\quad}
\renewcommand{\figurename}{\textbf{Рис.}}		%Чтобы вместо figure под рисунками писал "рис"
\renewcommand{\tablename}{\textbf{Таблица}}		%Чтобы вместо table над таблицами писал Таблица

%%% Теоремы
\theoremstyle{plain} % Это стиль по умолчанию, его можно не переопределять.
\newtheorem{theorem}{Теорема}[section]
\newtheorem{proposition}[theorem]{Утверждение}

\theoremstyle{definition} % "Определение"
\newtheorem{definition}{Определение}[section]
\newtheorem{corollary}{Следствие}[theorem]
\newtheorem{problem}{Задача}[section]

\theoremstyle{remark} % "Примечание"
\newtheorem*{nonum}{Решение}
\newtheorem{zamech}{Замечание}[theorem]

%%% Правильные мат. символы для русского языка
\renewcommand{\epsilon}{\ensuremath{\varepsilon}}
\renewcommand{\phi}{\ensuremath{\varphi}}
\renewcommand{\kappa}{\ensuremath{\varkappa}}
\renewcommand{\le}{\ensuremath{\leqslant}}
\renewcommand{\leq}{\ensuremath{\leqslant}}
\renewcommand{\ge}{\ensuremath{\geqslant}}
\renewcommand{\geq}{\ensuremath{\geqslant}}
\renewcommand{\emptyset}{\varnothing}

\usepackage{bm} %жирный греческий шрифт
%\usepackage{ulem}

\graphicspath{{images}}

\renewcommand{\baselinestretch}{1.3}

\title{programm}
\date{\today}
\usepackage[left=1.27cm,right=1.27cm,top=1.27cm,bottom=2cm]{geometry}


\newcommand{\jur}[1]{\textbf{Ж}: #1}
\newcommand{\mar}[1]{\textbf{М}: #1}
\begin{document}
	\begin{center}
		\large\textbf{Программа экзамена по дисциплине "Теоретическая механика"}
	\end{center}
	\begin{enumerate}
	\item \underline{Понятие положения равновесия} (\jur{153-155}). Критерий положения равновесия стационарной системы в терминах обобщенных сил. Случай потенциальных сил. Принцип  виртуальных перемещений.
	
	\item Теорема  Лагранжа-Дирихле  об  устойчивости положения равновесия консервативной  системы (\jur{168}, \mar{508-509}). Теоремы \underline{Ляпунова} (\jur{168}) и \textit{Четаева} (??) о неустойчивости положения равновесия консервативной системы (без доказательства).
	
	\item Влияние дополнительных гироскопических и \underline{диссипативных сил} (\jur{169}) на устойчивость консервативной системы. Обобщение теоремы Лагранжа-Дирихле при действии диссипативных сил с \underline{полной диссипацией} (\jur{169}).
	
	\item Теорема Ляпунова об устойчивости и неустойчивости по \underline{линейному приближению} (\jur{165}). 
	
	\item Необходимые условия устойчивости полинома (\jur{160}, лемма \textit{1}). \underline{Критерий  Рауса-Гурвица} (без доказательства) (\jur{159}).
	
	\item Теоремы второго метода Ляпунова об устойчивости, неустойчивости и асимптотической устойчивости.
	
	\item Теорема Четаева о неустойчивости.
	
	\item Теорема Барбашина-Красовского.
	
	\item Понятия о бифуркациях, дивергенции и флаттере. 
	
	\item Малые колебания консервативной системы в окрестности устойчивого положения  равновесия(\jur{171-173} \textit{одномерный}, \jur{174-180} \textit{многомерный}). Использование симметрии системы для нахождения мод колебаний (??).
	
	\item \underline{Свойства} амплитудных векторов в задаче о малых колебаниях консервативной системы, \underline{главные (нормальные) координаты} (\jur{178-180}).
	
	\item Действие внешней силы на линейную систему, движущуюся вблизи положения равновесия. Частотные характеристики (\jur{180-187}).
	
	\item Канонические  уравнения  Гамильтона (\jur{279}, \mar{292-294}).  Физический смысл функции Гамильтона в случае консервативной системы (\jur{280}, \mar{295-296}).
	
	\item Первые интегралы уравнений движения. Критерий первого интеграла, скобки Пуассона и их свойства (\jur{284-290}, \mar{345-346}).
	
	
	\item Понижение порядка гамильтоновой системы при наличии первых интегралов (\mar{338-339} ??). Уравнения Уиттекера (\mar{298-299}).
	
	\item Циклические координаты (\mar{337-338}). Обобщённо консервативные системы (??). Теорема Якоби-Пуассона (\mar{346-347}).
	
	\item Действие по Гамильтону (\mar{491-492}). Вариация  действия по Гамильтону в задаче с подвижными концами (\mar{493-497}).
	
	
	\item Принцип Гамильтона (\mar{484-491} \textit{очень подробно, с определениями}).
	
	\item Ковариантность уравнений Лагранжа при замене координат и времени.
	
	\item Теорема Э. Нетер (\jur{281}).
	
	\item Интегральные  инварианты Пуанкаре-Картана и Пуанкаре.
	
	\item Обратные  теоремы  теории  интегральных  инвариантов.
	
	\item Теорема Ли-Хуачжуна о множестве интегральных инвариантах первого порядка.
	
	\item Теорема Лиувилля: инвариантность фазового объема в системах с нулевой дивергенцией. Сохранение фазового объема в гамильтоновых системах (\mar{360}).
	
	\item Канонические преобразования (\mar{348-351}). Критерий каноничности преобразований (\mar{351-354}).
	
	\item Свободные канонические преобразования (\mar{360-363}).
	
	\item Преобразования функции Гамильтона при канонических преобразованиях (\jur{292-296}).
	
	\item Уравнение Гамильтона-Якоби (\mar{370-373}). Полный интеграл уравнения Гамильтона-Якоби (\jur{298-299}, \mar{371-372}). Разделение переменных (\mar{375-378}).
\end{enumerate}
\end{document}