\section{Основные понятия СТО}
\subsection{Преобразования Лоренца}
\begin{problem}
	Начало координат системы $K′$ движется со скоростью $V = (V_x, V_y)$ относительно системы $K$, а оси координат составляют со скоростью $\mathbf{V}$ те же самые углы, что и оси системы $K$. 
	Записать матрицу преобразований Лоренца от системы $K$ к системе $K′$ (а также обратного преобразования).
	Определить положение осей $(x′, y′)$ в системе $K$ в момент времени $t = 0$ по часам системы $K$.
	Указание: представить радиус-вектор в виде суммы параллельного и перпендикулярного скорости $\mathbf{V}$ векторов: $\mathbf{r} = \mathbf{r_\parallel} + \mathbf{r_\perp}$, где $\mathbf{r_\parallel} = (\mathbf{r} \cdot \mathbf{V})\mathbf{V}/V_2,\, \mathbf{r_\perp} = \mathbf{r} − (\mathbf{r} \cdot \mathbf{V}) \mathbf{V}/V_2$.
\end{problem}
