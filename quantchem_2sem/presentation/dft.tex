% Этот шаблон документа разработан в 2014 году
% Данилом Фёдоровых (danil@fedorovykh.ru) 
% для использования в курсе 
% <<Документы и презентации в \LaTeX>>, записанном НИУ ВШЭ
% для Coursera.org: http://coursera.org/course/latex .
% Исходная версия шаблона --- 
% https://www.writelatex.com/coursera/latex/5.1

\documentclass[t]{beamer}  % [t], [c], или [b] --- вертикальное выравнивание на слайдах (верх, центр, низ)
%\documentclass[handout]{beamer} % Раздаточный материал (на слайдах всё сразу)
%\documentclass[aspectratio=169]{beamer} % Соотношение сторон

% \usetheme{Berkeley} % Тема оформления
% \usetheme{Bergen}
\usetheme{Madrid}

\usecolortheme{seagull} % Цветовая схема
%\useinnertheme{circles}
%\useinnertheme{rectangles}

% \usetheme{HSE}

%%% Работа с русским языком
\usepackage{cmap}					% поиск в PDF
\usepackage{mathtext} 				% русские буквы в формулах
\usepackage[T2A]{fontenc}			% кодировка
\usepackage[utf8]{inputenc}			% кодировка исходного текста
\usepackage[english,russian]{babel}	% локализация и переносы

%% Beamer по-русски
\newtheorem{rtheorem}{Теорема}
\newtheorem{rproof}{Доказательство}
\newtheorem{rexample}{Пример}

%%% Дополнительная работа с математикой
\usepackage{amsmath,amsfonts,amssymb,amsthm,mathtools} % AMS
\usepackage{icomma} % "Умная" запятая: $0,2$ --- число, $0, 2$ --- перечисление

%% Номера формул
%\mathtoolsset{showonlyrefs=true} % Показывать номера только у тех формул, на которые есть \eqref{} в тексте.
%\usepackage{leqno} % Нумерация формул слева

%% Свои команды
\DeclareMathOperator{\sgn}{\mathop{sgn}}

%% Перенос знаков в формулах (по Львовскому)
\newcommand*{\hm}[1]{#1\nobreak\discretionary{}
{\hbox{$\mathsurround=0pt #1$}}{}}

%%% Работа с картинками
\usepackage{graphicx}  % Для вставки рисунков
\graphicspath{{images/}{images2/}}  % папки с картинками
\setlength\fboxsep{3pt} % Отступ рамки \fbox{} от рисунка
\setlength\fboxrule{1pt} % Толщина линий рамки \fbox{}
\usepackage{wrapfig} % Обтекание рисунков текстом

%%% Работа с таблицами
\usepackage{array,tabularx,tabulary,booktabs} % Дополнительная работа с таблицами
\usepackage{longtable}  % Длинные таблицы
\usepackage{multirow} % Слияние строк в таблице

%%% Программирование
\usepackage{etoolbox} % логические операторы

%%% Другие пакеты
\usepackage{lastpage} % Узнать, сколько всего страниц в документе.
\usepackage{soul} % Модификаторы начертания
\usepackage{csquotes} % Еще инструменты для ссылок
%\usepackage[style=authoryear,maxcitenames=2,backend=biber,sorting=nty]{biblatex}
\usepackage{multicol} % Несколько колонок

%%% Картинки
\usepackage{tikz} % Работа с графикой
\usepackage{pgfplots}
\usepackage{pgfplotstable}

\title{Теория функционала плотности}
% \subtitle{Документы и презентации в \LaTeX}
% \author{Данил Фёдоровых}
\date{}
% \institute[Высшая школа экономики]{Национальный исследовательский университет \\ <<Высшая школа экономики>>}

\begin{document}

\frame[plain]{\titlepage}	% Титульный слайд

\section{Общие положения}
% \subsection{Команда pause}
 
\begin{frame}
	\frametitle{\insertsection} 
	% \framesubtitle{\insertsubsection}
	\begin{itemize}
        \item Любое свойство системы взаимодействующих частиц может быть представлено как функционал от плотности в основном состоянии $n_{0} (\vec{r})$;
        \item Делаются предположения о виде обменно-корреляционного функционала;
	\end{itemize}
\end{frame}


\section{Основные теоремы}
\begin{frame}
	\frametitle{Формулировка теорем}
    \begin{rtheorem}[Хохенберга-Кона]
        \label{th1}
        Внешний потенциал $V_{ext} (\vec{r})$ определяется электронной плотностью основного состояния $n_{0} (\vec{r})$ с точностью до простой аддитивной константы.
    \end{rtheorem}
    \begin{rtheorem}
        \label{th2}
        Для заданного $V_{ext} (\vec{r})$ существует функционал $E[n] (\vec{r})$, принимающий минимальное значение на плотности $n_G (\vec{r})$, соответствующей основному состоянию. При этом $E[n_G]$ является энергией основного состояния.
    \end{rtheorem}
\end{frame}

\begin{frame}[shrink=7]
    \frametitle{Доказательство т.1}
        Предположим, что два потенциала $V_1 (\vec{r})$ и $V_2 (\vec{r})$ определяются одной и той же плотностью $n (\vec{r})$.
        Тогда имеется два Гамильтониана $H_1$ и $H_2$ с той же самой плотностью, но разными волновыми функциями $\Psi_1$ и $\Psi_2$ соответственно. Можем записать:
        \begin{multline}
            E_1^0 < \left\langle\Psi_{2}\left|H_{1}\right| \Psi_{2}\right\rangle=
            \left\langle\Psi_{2}\left|H_{2}\right| \Psi_{2}\right\rangle+\left\langle\Psi_{2}\left|H_{1}-H_{2}\right| \Psi_{2}\right\rangle= \\
            E_{2}^{0}+\int n(\vec{r})\left[V_{1}(\vec{r})-V_{2}(\vec{r})\right] d^3 r
        \end{multline}
        Аналогично:
        \begin{multline}
            E_2^0 < \left\langle\Psi_{1}\left|H_{2}\right| \Psi_{1}\right\rangle=
            \left\langle\Psi_{1}\left|H_{1}\right| \Psi_{1}\right\rangle+\left\langle\Psi_{1}\left|H_{2}-H_{1}\right| \Psi_{1}\right\rangle= \\
            E_{1}^{0}+\int n(\vec{r})\left[V_{2}(\vec{r})-V_{1}(\vec{r})\right] d^3 r
        \end{multline}
        Складывая два равенства, получаем $E_{1}^{0}+E_{2}^{0}<E_{2}^{0}+E_{1}^{0}$, что является противоречием. Предположение было неверным.
\end{frame}

\begin{frame}[shrink=7]
    \frametitle{Доказательство т.2}
    Рассматриваем ${n (\vec{r})}$, соответствующую гамильтониану в \textit{некотором} поле $V$ ($V$-представимая плотность). Тогда энергия является функционалом:
    \begin{equation}
        E[n]=T[n]+V_{ext}[n]+V_{e e}[n] \equiv \tilde{F}_{HK}[n]+ V_{ext}[n]
    \end{equation}
    \begin{equation}
        V_{ext}[n]=\int n(\vec{r}) V_{ext}(\vec{r}) d^3 r, \quad 
        \tilde{F}_{H K}[n]=T[n]+V_{e e}[n]
    \end{equation}
    Рассмотрим $n (\vec{r})$, соответствующую основному состоянию в поле $V$ (по т.1 такое поле найдется) и соответствующую в.ф. $\Psi$.
    На ней:
    \begin{equation}
        E^{0}_1 = E[n] = \left\langle\Psi\left|H\right| \Psi\right\rangle
    \end{equation}
    Возьмем другую произвольную $n' (\vec{r})$, соответствующую полю $V'$ и в.ф. $\Psi'$. 
    Имеем:
    \begin{equation}
        E^0_1=\left\langle\Psi\left|H\right| \Psi\right\rangle
        <
        \left\langle\Psi'\left|H\right| \Psi'\right\rangle=E_2^0
    \end{equation}
    Неравенство достигается потому, что $\Psi$ соответствует основному состоянию.
    На $n$ функционал достигает минимума и его значение является энергией основного состояния в потенциале $V$.

\end{frame}

\begin{frame}[shrink=20]{Уравения Кона-Шэма}
    % Основное предположение, не имеющее обоснования:
    % \begin{equation}
    %     T[n(\vec{r})]=-2 \frac{\hbar^{2}}{2 m} \sum_{i} \int \Psi_{i}^{*}(\vec{r}) \nabla^{2} \Psi_{i}(\vec{r}) d \vec{r}
    % \end{equation}
    Для системы $N$ невзаимодействующих электронов на $N$ орбиталях кинетическая энергия:
    \begin{equation}
        T_s [n(\vec{r})] = \sum_{i=1}^{N} \left\langle\varphi_{i}\left|-\frac{\hbar^2}{2m} \nabla^{2}\right| \varphi_{i}\right\rangle
    \end{equation}
    В случае взаимодействующих электронов выражение на $T[n]$ становится сложным. Но мы можем выделить из него $T_s[n]$, а остальное вынести в $V_{xc}$:
    \begin{equation}
        E[n] = T_s[n] + V_{ext}[n] + V_{ee}[n] + (T[n] - T_s[n])
    \end{equation}
    Выделяя $V_{ee}[n] = J [n] + \tilde{V}_{xc} [n] $, где $J$ - классический кулоновский потенциал:
    \begin{equation}
        J[n(\vec{r})]=\frac{1}{2} \int \frac{n\left(\vec{r}_{1}\right) n\left(\vec{r}_{2}\right)}{\left|\vec{r}_{1}-\vec{r}_{2}\right|} d \vec{r}_{1} d \vec{r}_{2}
    \end{equation}
    получим 
    \begin{equation}
        \label{eq:to_variate_E}
        E[n]=T_s[n]+V_{ext}[n]+J[n]+\underbrace{(\tilde
        {V}_{xc}[n] + T[n] - T_s[n])}_{V_{xc}}
    \end{equation}
    \textit{Предполагается}, что выделенный $V_{xc}$ является функционалом от $n$.
    Тогда $V_{xc}[n]$ называется \textit{обменно-корреляционным функционалом}.

\end{frame}
\begin{frame}{Уравнения Кона-Шэма}
    Необоходимо проварьировать функционал \ref{eq:to_variate_E} с учетом условия нормировки:
    \begin{equation}
        \int \delta n(\vec{r}) dr = 0, \quad n(\vec{r})=\sum_{i=1}^{N}\left|\varphi_{i}\right|^{2} 
    \end{equation}
    Последнее уравнение есть разложение плотности по волновым функциям в данном подходе.
    Метод Лагранжа дает:
    \begin{equation}
        \left\{\begin{array}{l}
{\left[-\frac{\hbar^{2}}{2 m} \nabla^{2}+V_{ext}(\vec{r})+
    \int \frac{n\left(r^{\prime}\right)}{\left|r-r^{\prime}\right|} d r^{\prime} + v_{xc}[n](\vec{r})\right] 
    \varphi_{i}(\vec{r})=
\varepsilon_{i} \varphi_{i}(\vec{r})} \\
\int \varphi_{i}^{*}(\vec{r}) \varphi_{i}(\vec{r}) d \vec{r}=1
\end{array}\right.
    \end{equation}
    где $v_{xc}[n](\vec{r}) = \cfrac{\delta V_{x c}[n]}{\delta n(\vec{r})}$ -- т.н. обменно-корреляционный потенциал.

    Определение $v_{xc}$ является основной проблемой метода.

\end{frame}

\section{Примеры обменно-корреляционных функционалов}

\begin{frame}[shrink=25]{\insertsection}
    \framesubtitle{Приближение локальной плотности (LDA)}
    Хохенберг и Кон показали, что если $n(\vec{r})$ изменяется очень медленно, тогда
    \begin{equation}
        E_{xc}^{LDA}[n]=\int n(\vec{r}) \varepsilon_{xc}(n) d^3 r
    \end{equation}
    где $\varepsilon_{xc}(n)$ -- обменная и корреляционная энергия одного электрона в однородном электронном газе с плотностью $n(\vec{r})$. Ее можно представить в виде суммы обменного и корреляционного взаимодействия:
    \begin{equation}
        \varepsilon_{xc}(n) = \varepsilon_{x}(n) + \varepsilon_{c}(n) 
    \end{equation}
    Энергию обменного взаимодействия можно представить в виде:
    \begin{equation}
        \varepsilon_{x}(n)=-\frac{3 q^{2}}{4}\left(\frac{3}{\pi}\right)^{1 / 3} \int(n(\vec{r}))^{4 / 3} d^{3} r
    \end{equation}
    а энергию корреляционного взаимодействия полагают равной
    \begin{equation}
        \varepsilon_{c}(n)=\varepsilon_{c}^{V W N}(n)
    \end{equation}
    При этом $\varepsilon_{c}(n)$ параметризуется путём интерполяции высокоточных расчётов, выполненных квантовым методом Монте-Карло.
\end{frame}

\begin{frame}[shrink=25]{\insertsection}
    \framesubtitle{Обобщённые градиентные функционалы (GGA)}
    Имеют вид:
    \begin{equation}
        E_{x c}^{G G A}[n]=\int f[n(\vec{r}), \nabla n(\vec{r})] d^3 r
    \end{equation}
    Одни из наиболее популярных GGA функционалов это PBE (Perdew, Burke, Ernzerhof) и BLYP (обменных функционал Becke и корреляционный функционал Lee, Yang, Parr). 
    \vspace{5mm}
    
    % Также существуют \textbf{гибридные функционалы}. Важнейшим функционалом ТФП является функционал B3LYP \cite{b3lyp}, объединяющий трёх-параметрический гибридный функционал Becke и корреляционный функционал Lee, Yang, Parr. B3 функционал добавляет Хартри-Фоковскую обменную энергию в функционал ТФП.

\end{frame}

\begin{frame}{Литература}
    \begin{thebibliography}{10}
        \bibitem{lectures} Лекции по квантовой химии весеннего семестра (2020)
        \bibitem{martin-book} Richard M.Martin <<Electronic Structure. Basic Theory and Practical Methods>>, Cambridge University Press (2004)
        \bibitem{b3lyp} P.J.Stephens, F.J.Devlin, C.F.Chabalowski, M.J.Frisch, J. Phys. Chem., \textbf{98}, 623 (1994) 
    \end{thebibliography}
\end{frame}

\end{document}
