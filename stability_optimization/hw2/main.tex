\documentclass[a4paper,12pt]{article}

%%% Работа с русским языком % для pdfLatex
\usepackage{cmap}					% поиск в~PDF
\usepackage{mathtext} 				% русские буквы в~фомулах
\usepackage[T2A]{fontenc}			% кодировка
\usepackage[utf8]{inputenc}			% кодировка исходного текста
\usepackage[english,russian]{babel}	% локализация и переносы
\usepackage{indentfirst} 			% отступ 1 абзаца

%%% Работа с русским языком % для XeLatex
%\usepackage[english,russian]{babel}   %% загружает пакет многоязыковой вёрстки
%\usepackage{fontspec}      %% подготавливает загрузку шрифтов Open Type, True Type и др.
%\defaultfontfeatures{Ligatures={TeX},Renderer=Basic}  %% свойства шрифтов по умолчанию
%\setmainfont[Ligatures={TeX,Historic}]{Times New Roman} %% задаёт основной шрифт документа
%\setsansfont{Comic Sans MS}                    %% задаёт шрифт без засечек
%\setmonofont{Courier New}
%\usepackage{indentfirst}
%\frenchspacing

%%% Дополнительная работа с математикой
\usepackage{amsfonts,amssymb,amsthm,mathtools}
\usepackage{amsmath}
\usepackage{icomma} % "Умная" запятая: $0,2$ --- число, $0, 2$ --- перечисление
\usepackage{upgreek}
%\usepackage{mathassents}

%% Номера формул
%\mathtoolsset{showonlyrefs=true} % Показывать номера только у тех формул, на которые есть \eqref{} в~тексте.

%%% Страница
\usepackage{extsizes} % Возможность сделать 14-й шрифт

%% Шрифты
\usepackage{euscript}	 % Шрифт Евклид
\usepackage{mathrsfs} % Красивый матшрифт

%% Свои команды
\DeclareMathOperator{\sgn}{\mathop{sgn}} % создание новой конанды \sgn (типо как \sin)
\usepackage{csquotes} % ещё одна штука для цитат
\newcommand{\pd}[2]{\ensuremath{\cfrac{\partial #1}{\partial #2}}} % частная производная
\newcommand{\abs}[1]{\ensuremath{\left|#1\right|}} % модуль
\renewcommand{\phi}{\ensuremath{\varphi}} % греческая фи
\newcommand{\pogk}[1]{\!\left(\cfrac{\sigma_{#1}}{#1}\right)^{\!\!\!2}\!} % для погрешностей

% Ссылки
\usepackage{color} % подключить пакет color
% выбрать цвета
\definecolor{BlueGreen}{RGB}{49,152,255}
\definecolor{Violet}{RGB}{120,80,120}
% назначить цвета при подключении hyperref
\usepackage[unicode, colorlinks, urlcolor=blue, linkcolor=blue, pagecolor=blue, citecolor=blue]{hyperref} %синие ссылки
%\usepackage[unicode, colorlinks, urlcolor=black, linkcolor=black, pagecolor=black, citecolor=black]{hyperref} % для печати (отключить верхний!)


%% Перенос знаков в~формулах (по Львовскому)
\newcommand*{\hm}[1]{#1\nobreak\discretionary{}
	{\hbox{$\mathsurround=0pt #1$}}{}}

%%% Работа с картинками
\usepackage{graphicx}  % Для вставки рисунков
\graphicspath{{images/}{images2/}}  % папки с картинками
\setlength\fboxsep{3pt} % Отступ рамки \fbox{} от рисунка
\setlength\fboxrule{1pt} % Толщина линий рамки \fbox{}
\usepackage{wrapfig} % Обтекание рисунков и таблиц текстом
\usepackage{multicol}

%%% Работа с таблицами
\usepackage{array,tabularx,tabulary,booktabs} % Дополнительная работа с таблицами
\usepackage{longtable}  % Длинные таблицы
\usepackage{multirow} % Слияние строк в~таблице
\usepackage{caption}
\captionsetup{labelsep=period, labelfont=bf}

%%% Оформление
\usepackage{indentfirst} % Красная строка
%\setlength{\parskip}{0.3cm} % отступы между абзацами
%%% Название разделов
\usepackage{titlesec}
\titlelabel{\thetitle.\quad}
\renewcommand{\figurename}{\textbf{Рис.}}		%Чтобы вместо figure под рисунками писал "рис"
\renewcommand{\tablename}{\textbf{Таблица}}		%Чтобы вместо table над таблицами писал Таблица

%%% Теоремы
\theoremstyle{plain} % Это стиль по умолчанию, его можно не переопределять.
\newtheorem{theorem}{Теорема}[section]
\newtheorem{proposition}[theorem]{Утверждение}

\theoremstyle{definition} % "Определение"
\newtheorem{definition}{Определение}[section]
\newtheorem{corollary}{Следствие}[theorem]
\newtheorem{problem}{Задача}[section]

\theoremstyle{remark} % "Примечание"
\newtheorem*{nonum}{Решение}
\newtheorem{zamech}{Замечание}[theorem]

%%% Правильные мат. символы для русского языка
\renewcommand{\epsilon}{\ensuremath{\varepsilon}}
\renewcommand{\phi}{\ensuremath{\varphi}}
\renewcommand{\kappa}{\ensuremath{\varkappa}}
\renewcommand{\le}{\ensuremath{\leqslant}}
\renewcommand{\leq}{\ensuremath{\leqslant}}
\renewcommand{\ge}{\ensuremath{\geqslant}}
\renewcommand{\geq}{\ensuremath{\geqslant}}
\renewcommand{\emptyset}{\varnothing}

\usepackage{bm} %жирный греческий шрифт
%\usepackage{ulem}

\graphicspath{{images}}


\usepackage{graphicx,xcolor,adjustbox,setspace, amsmath,pdfpages}

\newcommand{\resh}{\noindent\textit{Решение:}\\}

\newcounter{prim}
\newenvironment{prim}{%
	\addtocounter{prim}{1}
	\noindent{\\
		\textbf{\noindentПример \arabic{prim}\\}}%
}{\vspace{2mm}\\
\resh
}
\definecolor{orange}{rgb}{1, 0.7, 0.1}
%\usepackage{ulem}

\usepackage{bm} %жирный греческий шрифт

\newenvironment{psm}
{\left(\begin{smallmatrix*}[r]}
	{\end{smallmatrix*}\right)}

\newenvironment{pmatrixr}
{\begin{pmatrix*}[r]}
	{\end{pmatrix*}}

\newenvironment{solution}{%
    \textbf{Решение}%
}

\renewcommand{\figurename}{\textbf{Рис.}}		%Чтобы вместо figure под рисунками писал "рис"
\renewcommand{\tablename}{\textbf{Таблица}}		%Чтобы вместо table над таблицами писал Таблица


\title{ans2014}
\author{Кожарин Алексей}
\date{Nov 2021}
\usepackage[left=1.27cm,right=1.27cm,top=1.27cm,bottom=2cm]{geometry}

\begin{document}
	
\section{Задача №799}

    \subsection{Условие}
    Решить уравнение Риккати:

    \begin{equation}
        \label{eq:main_problem}
        \dot{K} + KP - S - TK + KQK = 0
    \end{equation}
    \begin{align*}
        P &= \begin{pmatrix}
            3 & -1 \\
            1 & 1
        \end{pmatrix}
        ,& Q &= \begin{pmatrix}
             1 & 0 \\
             1 & 0
        \end{pmatrix}
        ,& S &= \begin{pmatrix}
              4 & -1 \\
              0 & 0
        \end{pmatrix}
        ,& T &= \begin{pmatrix}
               4 & 0 \\
               0 & 1
        \end{pmatrix}
        ,& K_0 &= \begin{pmatrix}
                1 & 0 \\
                0 & 1
        \end{pmatrix}
    \end{align*}

    \subsection{Решение}
    Уравнение разбивается в виде:

    \begin{equation}
        \label{eq:solution_scheme}
        \begin{aligned}
            \frac{d\omega}{dt} &= \Omega \omega ,&
            \Omega &= \begin{pmatrix}
                P & Q \\
                S & T
            \end{pmatrix} ,&
            \frac{dx}{dt} &= Px + Qy ,&
            \frac{dy}{dt} &= Sx + Ty ,&
            y &= Kx
        \end{aligned}
    \end{equation}

    $\Omega$ имеет явный вид:

    \begin{equation}
        \Omega = \begin{pmatrix}
            3 & -1 & 1 & 0 \\
            1 & 1 & 1 & 0 \\
            4 & -1 & 4 & 0 \\
            0 & 0 & 0 & 1
        \end{pmatrix}
    \end{equation}

    Для уравнения $\cfrac{d\omega}{dt} = \Omega \omega$ имеем $\lambda_1 = 5$ , $\lambda_2 = 2$, $\lambda_3 = 1$, $\lambda_4 = 1$, что соответствует собственным векторам:

    \begin{align*}
        \lambda_1 &= 5 \Rightarrow \nu_1 = (1, 1, 3, 0)^T \\
        \lambda_2 &= 2 \Rightarrow \nu_2 = (-1, 2, 3, 0)^T \\
        \lambda_3 &= 1 \Rightarrow \nu_3 = (-1, -1, 1, 0)^T \\
        \lambda_4 &= 1 \Rightarrow \nu_4 = (0, 0, 0, 1)^T
    \end{align*}
    
    Это дает общие решения для уравнения $\cfrac{d\omega}{dt} = \Omega \omega$:
    \begin{equation}
        \label{eq:x_general}
        \begin{aligned}
            x_1 &= C_1 e^t \left(3 + 4 e^t + 5 e^{4t}\right) + C_2 e^t \left(-3 + 2e^t + e^{4t} \right) + C_3 e^t \left(-1 + e^{4t} \right) \\
            x_2 &= C_1 e^t \left(3 - 8e^t + 5e^{4t}\right) + C_2 e^t \left(-3 - 4e^t + e^{4t} \right) + C_3 e^t \left(-1 + e^{4t} \right) \\
            x_3 &= 3C_1 e^t \left(-1 - 4e^t + 5e^{4t} \right) + 3C_2 e^t \left( 1 - 2e^t + e^{4t} \right) + C_3 e^t \left( 1 + 3e^{4t} \right) \\
            x_4 &= C_4 e^t
        \end{aligned}
    \end{equation}

    Проведем замену $t' = t - T$, тогда начальные условия будут считаться в точке $t' = 0$, что сделает выкладки менее громоздкими.
    Вспомним также, что $y = Kx$ и $y = (x_3, x_4)^T$, $x = (x_1, x_2)^T$.
    Подставляя $t' = 0$, получим
    \begin{equation}
        \begin{aligned}
            \begin{pmatrix}
                x_3 \\
                x_4
            \end{pmatrix}
            &= K \begin{pmatrix}
                x_1 \\
                x_2
            \end{pmatrix}
            ,& t = T &\Rightarrow \begin{pmatrix} 4C_3 \\ C_4 \end{pmatrix}
             = K \begin{pmatrix} 12 C_1 \\ -6 C_2 \end{pmatrix}
        \end{aligned}
    \end{equation}

    Подставляя условие $K_0 = \begin{pmatrix}
        1 & 0 \\
        0 & 1
    \end{pmatrix}$, получим $C_3 = 3 C_1$, $C_4 = -6 C_2$.

    Эти соотношения на $C_i$ можно подставить в \eqref{eq:x_general}, что упростит выражение на $x_i$:

    \begin{equation}
        \begin{aligned}
            x_1 &= 2C_1 e^{2t'} \left( 2 + e^{3t'} \right) + C_2 e^{t'} \left( -3 + 2e^{t'} + e^{4t'} \right) \\
            x_2 &= 8C_1 e^{2t'} \left( -1 + e^{3t'} \right) + C_2 e^{t'} \left( -3 - 4e^{t'} + e^{4t'} \right) \\
            x_3 &= 12C_1 e^{2t'} \left( -1 + 2e^{3t'} \right) + 3C_2e^{t'} \left( 1 - 2e^{t'} + 4e^{4t'} \right) \\
            x_4 &= -6C_2 e^{t'}
        \end{aligned}
    \end{equation}

    Для поиска $K_{22}$ осуществим подстановку $x_1 = 0$, тогда получим
    \begin{equation}
        \label{eq:K_22_C}
        K_{22} = \cfrac{x_4}{x_2} = \frac{-6 C_2}{8C_1 e^{t'}\left( -1 + 3e^{t'} \right) + C_1 \left( -3 - 4e^{t'} + e^{4'} \right)}\,
    \end{equation}
    что с учетом $K_{22} (0) = 1$ даст соотношение
    \begin{equation}
        \label{eq:C_1_2}
        -3C_2 = 5C_1
    \end{equation}

    Подставляя \eqref{eq:C_1_2} в \eqref{eq:K_22_C}, получим в итоге:
    \begin{equation}
        \label{eq:K_22_final}
        K_{22} = \frac{10}{8e^{t'} \left( -1 + 3e^{t'} \right) - \frac{5}{3}\, \left( -3 - 4e^{t'} + e^{4t'} \right)   }\,
    \end{equation}

    Аналогичным образом получим остальные коэффициенты:
    \begin{equation}
        \label{eq:K_21_final}
        K_{21} = \frac{10}{2e^{t'} \left( 2 + e^{3t'} \right) - \frac{5}{3}\, \left( -3 + 2e^{t'} + e^{4t'} \right)   }\,
    \end{equation}
    \begin{equation}
        \label{eq:K_11_final}
        K_{11} = \frac{12 e^{t'} \left( -1 + 2 e^{3t'} \right) - 5 \left( 1 - 2 e^{t'} + 4e^{4t'} \right)}{2e^{t'} \left( 2 + e^{3t'} \right) - \frac{5}{3}\, \left( -3 + 2e^{t'} + e^{4t'} \right)   }\,
    \end{equation}
    \begin{equation}
        \label{eq:K_11_final}
        K_{12} = \frac{12 e^{t'} \left( -1 + 2 e^{3t'} \right) - 5 \left( 1 - 2 e^{t'} + 4e^{4t'} \right)}{8e^{t'} \left( -1 + 3e^{t'} \right) - \frac{5}{3}\, \left( -3 - 4e^{t'} + e^{4t'} \right)   }\,
    \end{equation}

    Это приводит к окончательному виду матрицы:

    \begin{equation}
        \label{eq:K_final}
        K(t) = \begin{pmatrix}
            \cfrac{12 e^{t-T} \left( -1 + 2 e^{3(t-T)} \right) - 5 \left( 1 - 2 e^{t-T} + 4e^{4(t-T)} \right)}{2e^{t-T} \left( 2 + e^{3(t-T)} \right) - \frac{5}{3}\, \left( -3 + 2e^{t-T} + e^{4(t-T)} \right)   }\,
            & 
            \cfrac{12 e^{t-T} \left( -1 + 2 e^{3(t-T)} \right) - 5 \left( 1 - 2 e^{t-T} + 4e^{4(t-T)} \right)}{8e^{t-T} \left( -1 + 3e^{t-T} \right) - \frac{5}{3}\, \left( -3 - 4e^{t-T} + e^{4(t-T)} \right)   }\,
            \\
            \cfrac{10}{2e^{t-T} \left( 2 + e^{3(t-T)} \right) - \frac{5}{3}\, \left( -3 + 2e^{t-T} + e^{4(t-T)} \right)   }\,
            &
            \cfrac{10}{8e^{t-T} \left( -1 + 3e^{t-T} \right) - \frac{5}{3}\, \left( -3 - 4e^{t-T} + e^{4(t-T)} \right)   }\,
        \end{pmatrix}
    \end{equation}

    Решение \eqref{eq:K_final} проверяется подстановкой в \eqref{eq:main_problem}.
    Выкладки достаточно объемны из-за громоздкого выражения на $\dot{K}$, поэтому были проведены в Wolfram Mathematica. Выражение \eqref{eq:K_final} является решением.
\end{document}
